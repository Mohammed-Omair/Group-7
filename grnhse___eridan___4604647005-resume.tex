


\documentclass[a4paper,10pt]{article}

% Package imports
\usepackage{latexsym}
\usepackage{xcolor}
\usepackage{float}
\usepackage{ragged2e}
\usepackage[empty]{fullpage}
\usepackage{wrapfig}
\usepackage{lipsum}
\usepackage{tabularx}
\usepackage{titlesec}
\usepackage{geometry}
\usepackage{marvosym}
\usepackage{verbatim}
\usepackage{enumitem}
\usepackage{fancyhdr}
\usepackage{multicol}
\usepackage{graphicx}
\usepackage{cfr-lm}
\usepackage[T1]{fontenc}
\usepackage{fontawesome}

% Color definitions
\definecolor{darkblue}{RGB}{0,0,0}

% Page layout
\setlength{\multicolsep}{0pt} 
\pagestyle{fancy}
\fancyhf{} % clear all header and footer fields
\fancyfoot{}
\renewcommand{\headrulewidth}{0pt}
\renewcommand{\footrulewidth}{0pt}
\geometry{left=1.4cm, top=0.8cm, right=1.2cm, bottom=1cm}
\setlength{\footskip}{5pt} % Addressing fancyhdr warning

% Hyperlink setup (moved after fancyhdr to address warning)
\usepackage[hidelinks]{hyperref}
\hypersetup{
    colorlinks=true,
    linkcolor=darkblue,
    filecolor=darkblue,
    urlcolor=darkblue,
}

% Custom box settings
\usepackage[most]{tcolorbox}
\tcbset{
    frame code={},
    center title,
    left=0pt,
    right=0pt,
    top=0pt,
    bottom=0pt,
    colback=gray!20,
    colframe=white,
    width=\dimexpr\textwidth\relax,
    enlarge left by=-2mm,
    boxsep=4pt,
    arc=0pt,outer arc=0pt,
}

% URL style
\urlstyle{same}

% Text alignment
\raggedright
\setlength{\tabcolsep}{0in}

% Section formatting
\titleformat{\section}{
  \vspace{-4pt}\scshape\raggedright\large
}{}{0em}{}[\color{black}\titlerule \vspace{-7pt}]

% Custom commands
\newcommand{\resumeItem}[2]{
  \item{
    \textbf{#1}{\hspace{0.5mm}#2 \vspace{-0.5mm}}
  }
}

\newcommand{\resumePOR}[3]{
\vspace{0.5mm}\item
    \begin{tabular*}{0.97\textwidth}[t]{l@{\extracolsep{\fill}}r}
        \textbf{#1}\hspace{0.3mm}#2 & \textit{\small{#3}} 
    \end{tabular*}
    \vspace{-2mm}
}

\newcommand{\resumeSubheading}[4]{
\vspace{0.5mm}\item
    \begin{tabular*}{0.98\textwidth}[t]{l@{\extracolsep{\fill}}r}
        \textbf{#1} & \textit{\footnotesize{#4}} \\
        \textit{\footnotesize{#3}} &  \footnotesize{#2}\\
    \end{tabular*}
    \vspace{-2.4mm}
}

\newcommand{\resumeProject}[4]{
\vspace{0.5mm}\item
    \begin{tabular*}{0.98\textwidth}[t]{l@{\extracolsep{\fill}}r}
        \textbf{#1} & \textit{\footnotesize{#3}} \\
        \footnotesize{\textit{#2}} & \footnotesize{#4}
    \end{tabular*}
    \vspace{-2.4mm}
}

\newcommand{\resumeSubItem}[2]{\resumeItem{#1}{#2}\vspace{-4pt}}

\renewcommand{\labelitemi}{$\vcenter{\hbox{\tiny$\bullet$}}$}
\renewcommand{\labelitemii}{$\vcenter{\hbox{\tiny$\circ$}}$}

\newcommand{\resumeSubHeadingListStart}{\begin{itemize}[leftmargin=*,labelsep=1mm]}
\newcommand{\resumeHeadingSkillStart}{\begin{itemize}[leftmargin=*,itemsep=1.7mm, rightmargin=2ex]}
\newcommand{\resumeItemListStart}{\begin{itemize}[leftmargin=*,labelsep=1mm,itemsep=0.5mm]}

\newcommand{\resumeSubHeadingListEnd}{\end{itemize}\vspace{2mm}}
\newcommand{\resumeHeadingSkillEnd}{\end{itemize}\vspace{-2mm}}
\newcommand{\resumeItemListEnd}{\end{itemize}\vspace{-2mm}}
\newcommand{\cvsection}[1]{%
\vspace{2mm}
\begin{tcolorbox}
    \textbf{\large #1}
\end{tcolorbox}
    \vspace{-4mm}
}

\newcolumntype{L}{>{\raggedright\arraybackslash}X}%
\newcolumntype{R}{>{\raggedleft\arraybackslash}X}%
\newcolumntype{C}{>{\centering\arraybackslash}X}%

% Commands for icon sizing and positioning
\newcommand{\socialicon}[1]{\raisebox{-0.05em}{\resizebox{!}{1em}{#1}}}
\newcommand{\ieeeicon}[1]{\raisebox{-0.3em}{\resizebox{!}{1.3em}{#1}}}

% Font options
\newcommand{\headerfonti}{\fontfamily{phv}\selectfont} % Helvetica-like (similar to Arial/Calibri)
\newcommand{\headerfontii}{\fontfamily{ptm}\selectfont} % Times-like (similar to Times New Roman)
\newcommand{\headerfontiii}{\fontfamily{ppl}\selectfont} % Palatino (elegant serif)
\newcommand{\headerfontiv}{\fontfamily{pbk}\selectfont} % Bookman (readable serif)
\newcommand{\headerfontv}{\fontfamily{pag}\selectfont} % Avant Garde-like (similar to Trebuchet MS)
\newcommand{\headerfontvi}{\fontfamily{cmss}\selectfont} % Computer Modern Sans Serif
\newcommand{\headerfontvii}{\fontfamily{qhv}\selectfont} % Quasi-Helvetica (another Arial/Calibri alternative)
\newcommand{\headerfontviii}{\fontfamily{qpl}\selectfont} % Quasi-Palatino (another elegant serif option)
\newcommand{\headerfontix}{\fontfamily{qtm}\selectfont} % Quasi-Times (another Times New Roman alternative)
\newcommand{\headerfontx}{\fontfamily{bch}\selectfont} % Charter (clean serif font)

\begin{document}
\headerfontiii

% Header
\begin{center}
    {\Huge\textbf{Mohammed Omair Mohiuddin}}
\end{center}
\vspace{-6mm}

\begin{center}
    \small{
    469-774-3039 | \href{mailto:mohammedomair07@gmail.com}{mohammedomair07@gmail.com} | 
    \href{https://mohdomair.netlify.app/}{mohdomair.netlify.app/}
    }
\end{center}
\vspace{-6mm}

\begin{center}
    \small{
    \socialicon{\faLinkedin} \href{https://www.linkedin.com/in/mohammed-omair/}{linkedin.com/in/mohammed-omair/} | 
    \socialicon{\faGithub} \href{https://github.com/Mohammed-Omair}{github.com/Mohammed-Omair}
    }
\end{center}

\vspace{-4mm}

\section{\textbf{Education}}
\vspace{-0.4mm}
\resumeSubHeadingListStart

\resumeSubheading
{University of Texas at Arlington}{Arlington, U.S}
{Master of Science in Computer Science}{May 2025}
\resumeItemListStart
\item GPA: 4.00/4.00
\resumeItemListEnd

\resumeSubHeadingListEnd
\vspace{-6mm}

\section{\textbf{Skills}}
\vspace{-0.4mm}
 \resumeHeadingSkillStart
  \resumeSubItem{Programming Languages:}
    {C, C++, Python, Rust}
  \resumeSubItem{Operating Systems:}
    {Linux}
  \resumeSubItem{Embedded Systems:}
    {Yocto}
  \resumeSubItem{Version Control:}
    {Git}
  \resumeSubItem{Certifications:}
    {{\href{https://www.credly.com/badges/4f2aa431-8ed9-4b9d-b438-3035ef2f9587/public_url}{Certified Kubernetes Administrator}}, {\href{https://www.credly.com/badges/d963c559-4088-49eb-8046-a8d314ef4a8a/public_url}{Linux Foundation Certified Systems Administrator}}, {\href{https://www.credly.com/badges/42af9d96-f66b-48ad-a771-5448d22b3b47/public_url}{HashiCorp Certified: Terraform Associate}}, {\href{https://www.credly.com/badges/c8ed5fc8-7ae0-4e7e-8b24-c97279288ab5/public_url}{AWS Certified Solutions Architect – Associate}}}
 \resumeHeadingSkillEnd

\section{\textbf{Experience}}
\vspace{-0.4mm}
\resumeSubHeadingListStart
  \resumeSubheading
      {Softwise Solutions}{Remote}
      {Embedded Software Engineer}{Apr 2021 - Apr 2023}
      \resumeItemListStart
        \item Implemented application-layer features and background services in \textbf{C++} and Linux user-space utilities in \textbf{C} for Pavillio on-device components (Cashe EVV v2, Cashe Communication), leading code reviews and maintaining collaborative development with \textbf{Git}
        \item Built reproducible custom Linux images and deployment artifacts for edge devices using \textbf{Yocto} and automated build scripts, enabling field-stable releases that integrated with Pavillio services (Cashe Exports, Salsa) across ARM-based gateways
        \item Developed and maintained kernel interfaces and drivers to support device I/O and secure data paths, using kernel logs and tracing to diagnose race conditions and optimize embedded performance during board bring-up
        \item Authored Python test cases and expanded automated hardware-in-the-loop test loops, integrating verification equipment (e.g., \textbf{Keysight}, \textbf{LitePoint}, \textbf{Viavi}) to validate RF/network behavior and regression-test embedded workflows
        \item Collaborated with hardware, QA, and backend teams to integrate embedded clients with Cashe Billing/Claims/Analytics APIs, implemented secure TLS communication and robust offline-first sync logic to improve reliability under intermittent connectivity
      \resumeItemListEnd 
\resumeSubHeadingListEnd
%End of experience section

\vspace{-6mm}

\section{\textbf{Projects}}
\vspace{-0.4mm}
\resumeSubHeadingListStart

\resumeProject
  {Embedded Linux Kernel Driver for RF Frontend}
  {C, Linux Kernel, Device Drivers, SPI/I2C, Device Tree, GDB, Yocto, Git}
  {}
  {}
\resumeItemListStart
  \item Implemented a Linux kernel driver for an SPI-controlled RF front-end: character device interface, IRQ handling, power management, and device-tree bindings; used printk/tracepoints and GDB/JTAG for low-level debugging during board bring-up
  \item Collaborated with hardware engineers to validate driver on target RU boards, created user-space test utilities and automated hardware test sequences, and integrated the driver into the platform Yocto layer with documented APIs
\resumeItemListEnd

\resumeProject
  {Yocto-Based Custom Linux Image for RU Platform}
  {Yocto, BitBake, Cross-compilation, Systemd, Kernel Configuration, Git}
  {}
  {}
\resumeItemListStart
  \item Designed and maintained a custom Yocto layer and BitBake recipes to produce reproducible Linux images for multiple RU targets, integrating kernel patches, device-tree overlays, and required user-space tools
  \item Automated multi-target builds and artifact management (cross-compile toolchains, rootfs, kernel), documented build/release steps, and enabled repeatable images for firmware bring-up and CI validation
\resumeItemListEnd

\resumeProject
  {Automated RF Test Framework}
  {Python, PyTest, SCPI, Keysight, LitePoint, Viavi, Serial/SSH, Test Automation}
  {}
  {}
\resumeItemListStart
  \item Built a Python-based test harness and PyTest suites to automate RF verification and regression tests (calibration, TX/RX chain checks) by driving Keysight/LitePoint instruments via SCPI and vendor APIs
  \item Implemented automated test loops, result parsing, and reporting to support nightly validation, reduced manual test time during board bring-up, and provided traceable logs for hardware/software debugging
\resumeItemListEnd

\resumeProject
  {gRPC-Based Multi-Threaded RU Control API}
  {C++, gRPC, Protocol Buffers, Multi-threading, Docker, Integration Testing}
  {}
  {}
\resumeItemListStart
  \item Developed a multi-threaded control-plane API in C++ using gRPC and Protocol Buffers to manage RU subsystems, implementing thread-safe command queues, health checks, and connection handling for simulated CU/DU interactions
  \item Containerized control and test services with Docker to create consistent integration testbeds, wrote end-to-end tests to validate concurrency, command ordering, and recovery scenarios during system-level debugging
\resumeItemListEnd

\resumeSubHeadingListEnd
%End of projects section
\vspace{-6mm}




\end{document}